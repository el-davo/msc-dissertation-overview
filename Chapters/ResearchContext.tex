\chapter{Research Context}
\lhead{\emph{Research Context}}
Micro-services are currently the hot new technology for the web, they allow us to break up our otherwise monolithic architecture into much smaller more focused services. This has a lot of benefits, such as reducing system size and complexity, release frequency and agility, data size, consistency, and complexity\cite{7742218}. 

There is one flaw with a micro-services architecture which I would like to attempt to address. Each service has its own logs which may contain information such as warnings, errors and stack-traces. When one service suddenly stops working it can be hard to determine the root cause of the failure. One of the reasons for this is that when someone is trying to diagnose an issue, it is possible they are looking at the logs of the wrong service. For example say we have 10 micro-services. Service 8 suddenly stops working. This could potentially be a knock on effect of something that happened with the other 9 services. The developer will naturally start with scanning the logs of the service that failed, and will have to work their way backward. This is both time consuming and can easily lead to a misdiagnosis.