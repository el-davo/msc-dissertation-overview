\chapter{Research Context}
\lhead{\emph{Research Context}}
Micro-services are currently the hot new technology for the web, they allow us to break up our otherwise monolithic architecture into much smaller more focused services. This has a lot of benefits, such as reducing system size and complexity, release frequency and agility, data size, consistency, and complexity\cite{7742218}. 

There is one flaw with a micro-services architecture which I would like to attempt to address. Each service has its own logs which may contain information such as warnings, errors and stack-traces. When one service suddenly stops working it can be hard to determine the root cause of the failure. One of the reasons for this is that when someone is trying to diagnose an issue, it is possible they are looking at the logs of the wrong service. For example say we have 10 micro-services. Service 8 suddenly stops working. This could potentially be a knock on effect of something that happened with the other 9 services. The developer will naturally start with scanning the logs of the service that failed, and will have to work their way backward. This is both time consuming and can easily lead to a misdiagnosis.

Log analysis is nothing new, however with the emergence of micro-services it has become far more challenging. Some of these challenges include

\begin{itemize}
  \item Logs from multiple services written in different languages. i.e. Node.js or Java
  \item Log streaming and log analysis in real time. 
  \item Store potentially Gigabytes of log data safely
  \item Auto-scaling
\end{itemize}

In my current organization we have a micro-service architecture. There are 100+ services written in multiple languages, including Node.js, Java, Python, Golang. We deploy all of our services onto Pivotal Cloud Foundry. Our current solution for log analysis is an ELK stack (Elastic-search, Log-stash and Kibana). This provides a nice interface to see the logs, however it has no opinions on where the logs come from or what information they hold. The user is on their own when trying to find a root cause to any issues.

There has already been a lot of research papers written on the subject log analysis. This paper written in 2017\cite{8067504} outlines the creation of a log analysis tool called POP that runs on an Apache Spark Cluster can accurately and quickly determine issues in logs. This tool has been open sourced at \href{https://github.com/logpai/logparser}{https://github.com/logpai/logparser}.

In this paper written in 2016\cite{7748933} There was again a proposal to use Apache Spark  along with Apache Flume for real time log analysis. 