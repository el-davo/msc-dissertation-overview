\chapter{Solution Design and Implementation}

For this project we have utilized the following technologies extensively.

\begin{itemize}
  \item NodeJS (Loopback 4, WS and GraphQL)
  \item Angular 7 with Highcharts
  \item Docker
  \item Kubernetes
  \item MySQL
  \item RabbitMQ
\end{itemize}

There are a few terms going forward that need to be defined. Below is an explanation of these terms

\begin{itemize}
  \item \emph{Pod} Is a group of one or more docker containers with shared storage and network running within Kubernetes.
  \item \emph{JWT} stand for JSON Web Token is an open standard \href{https://tools.ietf.org/html/rfc7519}{RFC 7519} for securely transmitting data between 2 places in JSON format. For the purposes of this project we will be using it for user authentication.
\end{itemize}


The project has been split up into a number of microservices to allow for maximum code readability and test-ability. The following microservices have been developed to make up the system.

\begin{itemize}
  \item ws-flare-ui
  \item ws-flare-user-api
  \item ws-flare-projects-api
  \item ws-flare-jobs-api
  \item ws-flare-orchestation-api
  \item ws-flare-cloud-foundry-monitor-api
  \item ws-flare-test-client
  \item ws-flare-cloudfoundry-monitor-client
  \item ws-flare-graphql
  \item ws-flare-gateway
  \item ws-flare-helm-chart
\end{itemize}

\subsection{ws-flare-ui}

This service is written in Angular 7. It holds all the code for displaying the user interface to a user. We also use Highcharts.js as a charting library for displaying results. The user interface is extensively tested using Jest testing framework from FaceBook.

\subsection{ws-flare-user-api}

Holds all user authentication information, such as email, user-name and passwords. For making authentication requests we are using JWT tokens to allow a user to query the system. Without this security any user could utilize the system to attack websites so it is important that we have this layer of security. This service is using MySQL for persisted storage. This service is tested using the mocha testing framework for NodeJS.

\subsection{ws-flare-projects-api}

The system allows users to create projects to logically separate each task. This is more for user convenience and adds to the overall UX aesthetics's of the system. All project and task information is stored within this service. This service uses MySQL as persisted storage. This service is also tested using the mocha testing framework for NodeJS.

\subsection{ws-flare-jobs-api}

This service holds information about currently running jobs. The system allows for running multiple jobs at the same time to allow to be truly embedded into continuous integration environments. This service is using MySQL for persisted storage. This service is tested using the mocha testing framework for NodeJS.

\subsection{ws-flare-orchestation-api}

This service handles the orchestation of jobs. It is listening on a rabbitMQ message queue for requests to start new jobs. Once it has received a new request it will calculate how many docker containers it needs to spin up to achieve the requested websocket load. It will then instruct Kubernetes to start the required amount of pods that it needs to start the stress test. It will also instruct kubernetes to start a pod designed specifically for monitoring Cloud Foundry applications. Once all pods have been started, this service will instruct all pods to start the test and for the Cloud Foundry monitor to start monitoring the specified applications. This service has no long term persisted storage mechanism and is testing using the mocha test framework for NodeJS.

\subsection{ws-flare-cloud-foundry-monitor-api}

This service stores information related to running applications on Cloud Foundry, such as memory and CPU at a particular point in time. This service is using MySQL for persisted storage. This service is tested using the mocha testing framework for NodeJS.

\subsection{ws-flare-test-client}

This is an application that can be spun up as a Kubernetes pod. The goal of this application is to simulate a number of websocket connections. A limit of 1000 connections per pod has been specified. For example if we have a test that requires a simulation of 5046 users then 6 ws-flare-test-client pods will be created. Five of those will simulate 1000 connections and the sixth will simulate 46 connections. Together they will simulate the full 5046 connections. For each of the connections established the connection information will be stored within the ws-flare-jobs-api service. The service will contain information on how many connections were successfully established, how many connections have dropped, and how many connections have not connected. We will also store the amount of time or latency it tool to connect to the websocket server. Once the stress test has completed this application will instruct Kubernetes to delete itself. Kubernetes will then proceed to remove the pod. This service has no long term persisted storage mechanism and is testing using the mocha test framework for NodeJS.

\subsection{ws-flare-cloudfoundry-monitor-client}

This is another application that is spun up as a Kubernetes pod. Once it is created it will immediately attempt to connect to the specified Cloud Foundry instance. There it will attempt to find the applications that the user requested to monitor. Once the correct applications have been found this application will send a message back to ws-flare-orchestation-api over a rabbitMQ message queue informing it that it has found the correct applications and is ready to start monitoring. The application will then wait for the orchestration API to create the necessary clients to simulate the Websockets. While the test is in progress the application will actively communicate with the Cloud Foundry instance to interrogate it for applications statistics such as memory and CPU usage. It will also get information from each instance of the application deployed on cloud foundry. Once all tests have completed this application will then shut itself down by instructing kubernetes to kill itself. Kubernetes will then remove the pod. This service has no long term persisted storage mechanism and is testing using the mocha test framework for NodeJS.

\subsection{ws-flare-graphql}

We chose to integrate GraphQL into this architecture due to its ability to easily query multiple services at the same time in one request. Due to the need to present results to the user with minimal latency we felt that this made sense, and very easy to integrate into Angular 7 using apollo-client. 

\subsection{ws-flare-gateway}

This is the API gateway to allow the user interface to communicate with the system. Using an API gateway provides a number of benefits, as everything is served through one endpoint or ip address. Currently the gateway has routes only the GraphQL server and the user interface server. The routes are outlined below

\begin{itemize}
  \item / routes to ws-flare-ui
  \item /graphql routes to ws-flare-graphql
\end{itemize}

The API gateway also provides a layer of security as the only way a user can interact with the system is through those 2 routes. It cannot communicate with any of the underlying services. The underlying technology of this gateway is an nginx server.

\subsection{ws-flare-helm-chart}

This is the helm chart for the entire application. Helm is the package manager for Kubernetes. With helm we can define our entire infrastructure as code. This also provides anyone with a Kubernetes cluster to install the application with one single command

\begin{minted}{bash}
helm install .
\end{minted}

Kubernetes will then work out all the routing between services using its internal DNS server. This makes Kubernetes a very attractive solution for distributing applications, and one of the main reasons Kubernetes was chosen for this project.

\subsection{Utilizing Kubernetes API to test at scale}

Stress testing as the name suggests puts a lot of stress on resources. Not only is this true for the system that is being tested, it is also true for the system performing the test. To eliminate resource drain on the system that is performing the test we need to be easily able to scale the test out for maximum throughput. Using a simple NodeJS script on a 16GB Memory intel core I7 labtop the maximum websockets that could be tested was peaking at around 10000 simultaneous websockets. After this amount it was observed that connections would either drop randomly or not connect at all to a websocket server. 

\begin{minted}{javascript}
var WebSocket = require('ws');

for (let i = 0; i < 100000; i++) {
    var ws = new WebSocket('ws://localhost:9002');

    ws.on('open', function open() {
        console.log('Opened socket ' + i);
        ws.send('PING');
    });

    ws.on('error', () => {
        console.log('Got error');
    });
}
\end{minted}

To maximize the number of websockets we can simulate we need a scalable system. This is the reason why Kubernetes was chosen for this project. With Kubernetes we can easily create new pods to perform the simulations. Kubernetes has a powerful rest API which we can perform a POST request on, passing it the required information it needs to create a new pod. The request looks like the example below.

\begin{minted}{typescript}
kubernetesClient
  .api
  .v1
  .namespaces('default')
  .pod
  .post({
    body: {
      kind: "Pod",
      apiVersion: "v1",
      metadata: {
        labels: {
          app: 'ws-flare-test-client'
        },
        name: 'ws-flare-test-client-abc1'
      },
      spec: {
        containers: [
          {
            name: 'ws-flare-test-client-abc1',
            image: 'wsflare/ws-flare-test-client',
            env: [],
            resources: {
              requests: {
                cpu: '100m'
              }
            }
          },
        ]
      }
    }
  });
  
\end{minted}

The above script will request that Kubernetes create a new Pod using the ws-flare-test-client docker container. We can also pass environment variables to this Pod. Using this API we can also specify how much CPU this pod is assigned. The above expression of 100m means that this pod will have one hundred millicpu. So if 1000m is the equivalent of one CPU then we can say that this Pod will be assigned one tenth of the avilable CPU cycles. From testing the application this is a good enough amount of CPU cycles to simulate 1000 websockets. 

If we consider that making the above request to Kubernetes will yield us one Pod which can simulate 1000 websockets, we can now start to think that if we make 10 requests and are given ten Pods, we can now simulate 10,000 websockets.

There are a number of options for obtaining a kubernetes cluster. Kubernetes can be deployed on an AWS Compute cluster using Amazons web-tools. Pivotal also has its own flavour of Kubernetes in the form of Pivotals Container Service (PKS). However in practive and while working on this project it was found that Google Kubernetes Engine (GKE) was one of the easiest kubernetes clusters to setup. It should also be noted that all of the above are paid services, and can become expensive depending on how much resources are assigned to a cluster. 

To setup a new cluster on GKE here are a few guidelines. The following assumes that you already have setup a GKE account setup.

\begin{figure}[!h]
  \centering
    \includegraphics[width=0.5\textwidth]{figures/gke-setup-1.png}
    \caption{Select Kubernetes Engine then select Clusters}
    \label{fig:https-handshake}
\end{figure}

\FloatBarrier

This will bring you into the kubernetes engine screen where you can then create your own cluster from scratch.

\begin{figure}[!h]
  \centering
    \includegraphics[width=0.8\textwidth]{figures/gke-setup-2.png}
    \caption{Select the resources to assign to this cluster}
    \label{fig:https-handshake}
\end{figure}

\FloatBarrier

Once you have selected your desired cluster resources, click Create and in a few minutes your cluster should be created. Once the cluster is setup you will be given the opportunity to connect to it. 

\begin{figure}[!h]
  \centering
    \includegraphics[width=0.8\textwidth]{figures/gke-setup-3.png}
    \caption{Click the connect button to get instructions on how to connect}
    \label{fig:https-handshake}
\end{figure}

\FloatBarrier

After clicking connect you will be given a command that you can run on your machine. Before running this command you will need to install the Google Cloud SDK tools. There is a quickstart guide available
\href{https://cloud.google.com/sdk/docs/quickstart-linux}{here}. Once you have installed the SDK tools then use the command that you have been given from the GKE console and you should be able to connect.

\subsection{Providing a scripting interface}

To ensure that it is possible to test a websocket server in a number of scenarios, the ws-flare platform allows users to essentially script how they want the test to progress. This enables the user to simulate a lot more scenarios then just simulating a number of connections at a time. When users create a new task with the platform they will be offered the opportunity to add a script. The script itself is a JSON array. An example of a valid script is below.

\begin{minted}{JSON}
[
    {
        "start": 0,
        "timeout": 55,
        "totalSimulators": 10000,
        "target": "wss://ws-flare-test-server.cfapps.io:4443"
    },
    {
        "start": 60,
        "timeout": 30,
        "totalSimulators": 5000,
        "target": "wss://ws-flare-test-server.cfapps.io:4443"
    },
    {
        "start": 120,
        "timeout": 60,
        "totalSimulators": 15000,
        "target": "wss://ws-flare-test-server.cfapps.io:4443"
    }
]
\end{minted}

In this scenario, we will initially be simulating 10000 connection o a websocket server. After 55 seconds, those 10000 connection will disconnect and one minute into the test another 5000 connection will attempt to connect to the server and disconnect after 30 seconds. Finally 2 minutes into the test 15000 connection will be simulated for one minute. Scripting is a very powerful feature of the stress testing platform. 

One scenario we hope to detect with scripting is the ability for a websocket server to actively clean up after itself when users disconnect. One scenario we found recently where this would have been applicable is having a websocket server backed by a Redis database. The role of the websocket server was to send notifications. Whenever a new websocket connected to the server the websocket server would create a new connection to the Redis database to wait for notifications. We noticed ever hour or so the application would crash and restart automatically. After some investigation it was found that the Redis connections were never dropped when a user disconnected their websocket. Over time the websocket server would simply be starved of memory and crash. With this scripting ability we can actively test for this kind of scenario, among many other scenarios.

\subsection{Pivotal Cloud Foundry Monitoring}

As mentioned already, Cloud Foundry exposes a powerful API. As a ws-flare test is in progress the platform will actively attempt to gain valuable metrics from this API. The first thing that happens when the test begins is that the system will attempt to log into cloud foundry using the user provided email and password. The login API endpoint request code is below

\begin{minted}{typescript}
async login(authorization_endpoint: string): Promise<Token> {
    const token = await post('https://api.run.pivotal.io
/oauth/token', {
            headers: {
                Authorization: 'Basic Y2Y6',
                'Content-Type': 'application/x-www-form-urlencoded'
            },
            json: true,
            form: {
                grant_type: 'password',
                client_id: 'cf',
                username: this.cfUser,
                password: this.cfPass
            }
        });

    return token.content as any;
}
\end{minted}

Using this code we can get an access token that can be used to query the endpoints we need. A few important points, we need to set Basic Y2Y6 as an Authorization header and the correct Content-Type. It is also necessary to specify the grant\_type which in this case is password. Cloud Foundry offers a number of grant types such as password and oauth. 

Cloud Foundry is split up into 3 logical entities. These are

\begin{itemize}
  \item Organizations
  \item Spaces
  \item Applications
\end{itemize}

Organizations represent a department in a company. The organization is the entity that is billed for the resources used on Cloud Foundry. On PWS (Pivotal Web Services) which is Pivotals cloud offering of Cloud Foundry this is measured in both CPU usage and Memory usage over a period of a month.

Spaces represent different environments. Spaces make environment repeat-ability very simple. For example you could have a DEV, STAGING and PROD environment all with the same applications deployed and utilizing the same resources, however they are used for very different things. The DEV environment can be used by developers for testing code changes. The STAGING environment can be used by a quality assurance team to test any new features before reaching the PROD environment. the PROD environment is the last stop and this is the environment that the end user will be connecting to.

Just like there can be multiple spaces in an organization, there can also be multiple applications deployed in a space. The applications themselves are the applications that a developer write. Applications on Cloud Foundry run in droplets, which are a kin to docker containers. Droplets require a user to specify a buildpack. Buildpacks are runtime environments for an application. Cloud Foundry supports many buildpacks for many different languages such as Java and NodeJS. To deploy an application to cloud foundry a user only needs to have an account and a manifest.yml file in the root of their project. A typical manifest file for a JAVA project might look like.

\begin{minted}{YAML}
name: cardsec
instances: 1
memory: 1024M
path: build/libs/cardsec-0.0.1-SNAPSHOT.jar
buildpack: java_buildpack
\end{minted}

The name represents the name of the application on cloud foundry. The instances are the number of running instances of the application, this allows users to easily scale up their applications under heavy load. The memory field is the total amount of memory that this app is allowed to consume. If this limit is exceeded then an application will likely crash with an out of memory exception. The path is the path to the generated jar file for this Java project and the buildpack is the desired run time environment, which in this case is Java.

Once ws-flare has gained access to cloud foundry we can now use the token to interrogate applications running within a specified space. Each running application has a unique GUID assigned to it. To find these unique ids we first need to find which organization and space the applications are running in. We first retrieve the list of organizations using the code below

\begin{minted}{typescript}
var response = json('https://api.run.pivotal.io/v2/organizations', {
    headers: {
        Authorization: token.token_type + ' ' + token.access_token,
        Accept: "application/json"
    }
});
\end{minted}

This will yield a list of organization running in Cloud Foundry that the user has access to. Each organization also has a unique GUID. Once we find the correct GUID we can then use this to search for the correct space by using the code below.

\begin{minted}[breaklines]{typescript}
var response = json('https://api.run.pivotal.io/v2/organizations/' + orgId + '/spaces', {
    headers: {
        Authorization: token.token_type + ' ' + token.access_token,
        Accept: "application/json"
    }
});
\end{minted}

This will yield a list of spaces. Again each space has a unique GUID which we can use to get all the applications running within a space. Finally to get a list of applications running within a space we can use the following code