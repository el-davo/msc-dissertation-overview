\begin{abstract}

Microservice architectures have seen widespread adoption in industry in recent years \cite{fowlerOnMicroservices}. They enable the separation of critical components into loosely coupled services, this, in turn, enables teams to adopt agile methodologies such as DevOps, continuous delivery, and test-driven development for validation.

A lesser used form of validating microservices is stress testing, which tests the robustness and error handling of systems under heavy load. Stress testing is often an afterthought until it becomes a critical issue. A reason this can be unappealing for teams to implement is due to the complexities required to set up a test infrastructure as well as the financial cost of hardware and test planning when compared to unit testing or integration testing.

This project demonstrates an approach to stress testing that validates and monitors services within microservice architectures by actively monitoring the underlying services of an application while testing is in progress, to allow developers gain a greater understanding of how increased traffic is affecting their systems. This framework currently stress tests web socket protocols, however, in the future this can be expanded for testing other protocols.

\end{abstract}